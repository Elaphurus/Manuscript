\section{AI4PA: Artificial intelligence for program analysis}

\subsection{AI4NLP}

自然语言处理(Natural Language Processing,NLP)主要研究如何让计算机能够理解、生成人类语言,包括早期基于统计方法的文本分类、信息检索等研究,以及近年来基于深度神经网络的机器翻译、问答系统、情感分析等研究。

\subsubsection{原理}

NLP 的原理包含多方面,如语言学、统计学、计算机科学等,它涉及对语言的结构、语义、语法、语用等方面的研究,以及对大规模语料库的统计分析和模型建立。

\paragraph{语言模型}
字符序列的概率分布就是一个最简单的语言模型。例如,在一个网页中,我们可以统计得到 $P(“the”)=0.027$。长度为 n 的书写符号序列称为 n-元组(n-gram)(gram 是希腊语表示“书写”或“字母”的单词的的词根)。n 个字符序列上的概率分布就称为 n 元字符模型。单词模型、音节模型同理。

n 元模型可以定义为 n-1 阶 Markov 链,即 $c_i$ 的概率仅和 $c_{i-1}$、$c_{i-2}$,...,$c_{i-(n-1)}$ 有关。

对于一个包含 100 个字符的语言的 3 元字符模型,$c_i$ 的概率有 100 万项参数。这些参数可以通过计数的方式,对包含上千万字符的文本集合进行统计得到。我们把文本集合称作\textit{语料库}。

作为对语言模型的改进,\textit{平滑}是一种调整低频计数的概率的过程,使得语料库中出现概率为零的序列会被赋予一个很小的非零概率值(其他数值会小幅下降以使概率和仍为 1)。

\paragraph{特征向量}

与线性代数无关,机器学习中的特征向量是一个特征值向量。假设现在我们有一个\textit{文本分类}的自然语言处理任务,目标是判断一封邮件是否是垃圾邮件。使用 1 元单词模型,语言模型中包含 10 万个单词,那么特征就是词汇表中的单词,即一个 100000 元组,特征的值就是每个单词在邮件中出现的次数,即特征向量 X 长度为 10 万。这种一元的表示形式被称为\textit{词袋}模型。这样我们就可以将垃圾邮件的特征向量用于监督学习。例如,“for cheap”、“You can buy”等 n 元单词序列很像是垃圾邮件的特征。

\subsubsection{流程}

NLP 的流程一般包含以下几个步骤:

\begin{enumerate}
\item \textbf{数据收集和预处理}:获取和清洗原始语言数据,如语料库
\item \textbf{分析和词法分析}:将原始数据转换为适合模型输入的格式
\item \textbf{特征提取}:将文本转换为计算机可以处理的向量形式
\item \textbf{模型训练}:训练 NLP 模型
\item \textbf{模型评估}:使用验证数据集评估模型的性能
\item \textbf{模型应用}:将模型应用于实际问题
\end{enumerate}

\subsection{程序作为文本}

程序分析指对计算机程序进行自动化地处理,以确认或发现某些程序性质,如正确性、安全性、性能等,程序分析的结果可以用于编译优化、漏洞检查等任务。经典的程序分析技术包括数据流分析、符号执行等。近年来机器学习也被用于一些程序分析任务。

\begin{lstlisting}[frame=none,numbers=none]
#include <stdlib.h>
#include <string.h>
int main()
{
  char *str;
  str = (char *) malloc(15);
  strcpy(str, "abc");
  // free(str);
  return(0);
}
\end{lstlisting}

假设我们要检查如上代码所示的内存泄露漏洞,语言模型是 C 语言的表达式,包含变量声明、库函数调用、返回等,语料库是正确的程序,那么在 n 元表达式模型中(假设 n 足够大),库函数调用 \texttt{malloc} 和 \texttt{free} 的特征值应当是相等的,否则就可能存在内存泄露。

我们再看一个例子,类型检查(或类型推断)是编译器的重要组件。

\begin{lstlisting}[frame=none,numbers=none]
// type error 1, C
char x = 'a';
x = "abc";

// type error 2, Java
switch (true) {
  case false: break;
}

// type error 3, Swift
class C {}
Class D {}
let x = true ? C() : D()
\end{lstlisting}

如上三个例子都包含类型错误。例 1 中,x 赋值为字符串,与其声明的 char 类型不符。例 2 中,switch 不接收布尔类型的变量。例 3 中,因为 C 和 D 没有公共父类型,导致三元运算符的类型不匹配。

简单地,类型检查以一个代码块和一个预期类型作为输入,递归下降地对代码块的子节点进行类型检查,最后输出一个布尔值表示是否存在类型错误。例如,

\begin{lstlisting}[frame=none,numbers=none]
typeCheck(Add(1, '2'), Int)
  = typeCheck(1, Int) && typeCheck('2', Int)
  = true && false
  = false
\end{lstlisting}

使用机器学习方法,n 元字符模型可以学到字面量如 1、\texttt{'2'}、\texttt{"abc"} 等的类型,n 元表达式模型可以学到赋值语句、switch 表达式、三元运算符等的类型约束。

甚至对于一些经典类型检查和推断方法中的复杂情形,比如范型类型的类型推断,机器学习方法可以学到一些启发式的知识。例如,对于 \texttt{isEqual<t1, t2>},n 元字符模型可以学到名为 \texttt{isXXX} 的函数大概率返回布尔类型。

\subsection{AI4PA}

程序语言相对于自然语言有两个特点:(1)更强的结构化,如嵌套和跳转;(2)作为形式语言,有精确定义的语言模型,如语法和语义规则。
