\section{[Nobel Literature Prize 1901] Sully Prudhomme}

``\textit{in special recognition of his poetic composition, which gives evidence
  of lofty idealism, artistic perfection and a rare combination of the qualities
  of both heart and intellect}''

“\textit{特此表彰他的诗作,证明了崇高的理想主义、艺术的完美以及心灵和智慧的罕见结合}”

\subsection{Le Vase Bris\'e (The Broken Vase,破碎的花瓶)}

\begin{multicols}{2}
The vase where this verbena’s dying          \par
Was cracked by a lady’s fan’s soft blow.     \par
It must have been the merest grazing:        \par
We heard no sound. The fissure grew.         \\

The little wound spread while we slept,      \par
Pried deep in the crystal, bit by bit.       \par
A long, slow marching line, it crept         \par
From spreading base to curving lip.          \\

The water oozed out drop by drop,            \par
Bled from the line we’d not seen etched.     \par
The flowers drained out all their sap.       \par
The vase is broken: do not touch.            \\

The quick, sleek hand of one we love         \par
Can tap us with a fan’s soft blow,           \par
And we will break, as surely riven           \par
As that cracked vase. And no one knows.      \\

The world sees just the hard, curved surface \par
Of a vase a lady’s fan once grazed,          \par
That slowly drips and bleeds with sadness.   \par
Do not touch the broken vase.

%%
马鞭草正在枯萎的花瓶       \par
被妇人的羽扇轻轻吹裂       \par
那是最轻柔的鞭笞          \par
裂缝在无声中生长          \\

伤口在夜晚蔓延            \par
一点点击碎水晶般的身躯     \par
漫长而缓慢的细线          \par
爬满腰身与脖颈            \\

水一滴滴地流出            \par
在我们看不到的伤口         \par
马鞭草流干了它的眼泪       \par
不要触碰,破碎的花瓶       \\

所爱之人的轻柔细手         \par
像羽扇的微风吹拂           \par
我们不可避免地分离         \par
一如那破碎的花瓶           \\

世人看到坚硬的外壳         \par
被轻柔的羽扇拂过           \par
却不见那缓缓滴落的心血与哀伤 \par
不要触碰,破碎的花瓶
\end{multicols}

\begin{thebibliography}{9}
\bibitem{brokenvase}
F. (2020) \emph{一樣的花瓶,不一樣的普呂多姆和納蘭性德}, \url{https://www.soleil-noir.me/blog/broken-vase/}.
\end{thebibliography}
