\section{Semantics engineering}

\subsection{Theory of programming languages}

\paragraph{Calculus} is a logic for calculating with the terms of the language.
For example,

\hspace{\parindent}
e = x | $\lambda$ x.e | (e e)

Extensions with primitive data:

\hspace{\parindent}
e = x | $\lambda$ x.e | (e e) | tt | ff | (if e e e)

External interpretation functions ($\delta$):

\hspace{\parindent}
(if tt e e') if-tt e

\hspace{\parindent}
(if ff e e') if-ff e'

\paragraph{Semantics} is a system for determining the value of a program.

\paragraph{Reduction} is a relation on terms:

\hspace{\parindent}
(($\lambda$ x.e) e') beta e[x = x'] (e with x replaced by e')

\hspace{\parindent}
(($\lambda$ x.e) e') beta [e'/x]e (substitution e' for x in e)

\paragraph{Equational system} is defined with three properties:

For any relation R,

\begin{multicols}{3}
\textit{reflexivity}
$\frac{\text{e R e'}}{\text{e e'' R e' e''}}$

\textit{symmetry}
$\frac{\text{e R e'}}{\text{e'' e R e'' e'}}$

\textit{transitivity}
$\frac{\text{e R e'}}{\text{$\lambda$ x.e R $\lambda$ x.e'}}$
\end{multicols}

With an equational system, we can prove such facts as

\hspace{\parindent}
e (Y e) = (Y e)

meaning every single term has a \textit{fixpoint}.

\asterism

In Plotkin's theory of programming languages, a semantic is a relation
\textit{eval} from programs to values:

\hspace{\parindent}
eval : Program $\times$ Value

\hspace{\parindent}
def e eval v iff e = v

We get a \textit{specification} of an interpreter after proving that eval is a
function.

\hspace{\parindent}
eval : Program $\rightarrow$ Value

\hspace{\parindent}
eval(e) = v

Prove that the calculus satisfies a standard reduction property. This gives us a
second semantic.

\hspace{\parindent}
eval-standard : Program $\rightarrow$ Value

\hspace{\parindent}
def eval-standard(e) = v iff e standard reduces to v

Curry-Feys's \textit{standard reduction} is a strategy for the lambda calculus,
that is, a function that picks the next reducible expression (called
\textit{redex}) to reduce. Plotkin specifically uses the leftmost-outermost
strategy.

Plotkin adds the \textit{truth} to the specification.

\hspace{\parindent}
def e $\sim$ e' iff placing e and e' into any context yields programs that

\hspace{\parindent}
produce the same observable behavior according to eval

\subsection{Redex}

Redex is a scripting language and set of associated tools
supporting the conception, design, construction, and testing of semantic systems
such as programming languages, type systems, program logics, and program
analyses. As a scripting language, it enables an engineer to create executable
specifications of common semantic elements such as grammars, reduction
relations, judgments, and metafunctions; the basic elements of formal systems.

\subsection{Syntax}

\begin{lstlisting}[frame=none,numbers=none]
; syntax trees
(define-language Lambda
  (e ::= x
         (lambda (x ...) e)
         (e e ...))
  (x ::= variable-not-otherwise-mentioned))

; instances
(define e1 (term y))
(define e2 (term (lambda (y) y)))
(define e3 (term (lambda (x y) y)))
(define e4 (term (,e2 ,e3)))

; a predicate that tests membership
(define lambda? (redex-match? Lambda e))

; language tests formulations
(test-equal (lambda? e1) #true)
(test-equal (lambda? e2) #true)
(test-equal (lambda? e3) #true)
(test-equal (lambda? e4) #true)

(define eb1 (term (lambda (x x) y)))
(define eb2 (term (lambda (x y) 3)))

(test-equal (lambda? eb1) #true)
(test-equal (lambda? eb2) #false)
\end{lstlisting}

\subsection{Metafunction}

A metafunction is a function on terms of a specific language.

\begin{lstlisting}[frame=none,numbers=none]
; are the identifiers in the given sequence unique?
; extended Kleene patterns: (lambda (x_!_ ...) e)

(module+ test
  (test-equal (term (unique-vars x y)) #true)
  (test-equal (term (unique-vars x y x)) #false))

(define-metafunction Lambda
  ; a Redex contract with patterns
  unique-vars : x ... -> boolean
  [(unique-vars) #true]
  [(unique-vars x x_1 ... x x_2 ...) #false]
  [(unique-vars x x_1 ...) (unique-vars x_1 ...)])

(module+ test
  (test-results))

; (subtract (x ...) x_1 ...) removes x_1 ... from (x ...)

(module+ test
  (test-equal (term (subtract (x y z x) x z)) (term (y))))

(define-metafunction Lambda
  subtract : (x ...) x ... -> (x ...)
  [(subtract (x ...)) (x ...)]
  [(subtract (x ...) x_1 x_2 ...)
   (subtract (subtract1 (x ...) x_1) x_2 ...)])

(module+ test
  (test-results))

; (subtract1 (x ...) x_1) removes x_1  from (x ...)

(module+ test
  (test-equal (term (subtract1 (x y z x) x)) (term (y z))))

(define-metafunction Lambda
  subtract1 : (x ...) x -> (x ...)
  [(subtract1 (x_1 ... x x_2 ...) x)
   (x_1 ... x_2new ...)
   (where (x_2new ...) (subtract1 (x_2 ...) x))
   (where #false (in x (x_1 ...)))]
  [(subtract1 (x ...) x_1) (x ...)])

(define-metafunction Lambda
  in : x (x ...) -> boolean
  [(in x (x_1 ... x x_2 ...)) #true]
  [(in x (x_1 ...)) #false])

(module+ test
  (test-results))
\end{lstlisting}

\subsection{Scope}

To specify the scope, a free-variables function specifies which language
constructs bind and which one don't.

\begin{lstlisting}[frame=none,numbers=none]
; (fv e) computes the sequence of free variables of e
; a variable occurrence of x is free in e
; if no (lambda (... x ...) ...) dominates its occurrence

(module+ test
  (test-equal (term (fv x)) (term (x)))
  (test-equal (term (fv (lambda (x) x))) (term ()))
  (test-equal (term (fv (lambda (x) (y z x)))) (term (y z))))

(define-metafunction Lambda
  fv : e -> (x ...)
  [(fv x) (x)]
  [(fv (lambda (x ...) e))
   (subtract (x_e ...) x ...)
   (where (x_e ...) (fv e))]
  [(fv (e_f e_a ...))
   (x_f ... x_a ... ...)
   (where (x_f ...) (fv e_f))
   (where ((x_a ...) ...) ((fv e_a) ...))])
\end{lstlisting}

\textit{$\alpha$ equivalence} is a realtion that virtually eliminates variables
from phrases and replaces them with arrows to their declarations. In
lambda calculus-based languages, this transformation is often a part of the
compiler, called the \textit{static-distance} phase.

\begin{lstlisting}[frame=none,numbers=none]
; (sd e) computes the static distance version of e

(define-extended-language SD Lambda
  (e ::= ....
         (K n n)
         n)
  (n ::= natural))

(define sd1 (term (K 1 1)))
(define sd2 (term 1))

(define SD? (redex-match? SD e))

(module+ test
  (test-equal (SD? sd1) #true)
  (test-equal (SD? sd2) #true))


(define-metafunction SD
  sd : e -> e
  [(sd e_1) (sd/a e_1 ())])

(module+ test
  (test-equal (term (sd/a x ())) (term x))
  (test-equal (term (sd/a x ((y) (z) (x)))) (term (K 2 0)))
  (test-equal (term (sd/a ((lambda (x) x) (lambda (y) y)) ()))
              (term ((lambda () (K 0 0)) (lambda () (K 0 0)))))
  (test-equal (term (sd/a (lambda (x) (x (lambda (y) y))) ()))
              (term (lambda () ((K 0 0) (lambda () (K 0 0))))))
  (test-equal (term (sd/a (lambda (z x) (x (lambda (y) z))) ()))
              (term (lambda () ((K 0 1) (lambda () (K 1 0)))))))

(define-metafunction SD
  sd/a : e ((x ...) ...) -> e
  [(sd/a x ((x_1 ...) ... (x_0 ... x x_2 ...) (x_3 ...) ...))
   ; bound variable
   (K n_rib n_pos)
   (where n_rib ,(length (term ((x_1 ...) ...))))
   (where n_pos ,(length (term (x_0 ...))))
   (where #false (in x (x_1 ... ...)))]
  [(sd/a (lambda (x ...) e_1) (e_rest ...))
   (lambda () (sd/a e_1 ((x ...) e_rest ...)))]
  [(sd/a (e_fun e_arg ...) (e_rib ...))
   ((sd/a e_fun (e_rib ...)) (sd/a e_arg (e_rib ...)) ...)]
  [(sd/a e_1 any)
   ; a free variable is left alone
   e_1])
\end{lstlisting}

Steps of the last formulation:

\begin{lstlisting}[frame=none,numbers=none]
   (sd/a (lambda (z x) (x (lambda (y) z))) ())
-> (lambda () (sd/a (x (lambda (y) z)) ((z x))))
-> (lambda () ((sd/a x ((z x))) (sd/a (lambda (y) z) ((z x)))))
-> (lambda () ((K 0 1) (lambda () (sd/a z ((y) (z x))))))
-> (lambda () ((K 0 1) (lambda () (K 1 0))))
\end{lstlisting}

$\alpha$ equivalence:

\begin{lstlisting}[frame=none,numbers=none]
; (=`$\alpha$` e_1 e_2) determines whether e_1 and e_2 are `$\alpha$` equivalent

(module+ test
  (test-equal (term (=`$\alpha$` (lambda (x) x) (lambda (y) y))) #true)
  (test-equal (term (=`$\alpha$` (lambda (x) (x 1)) (lambda (y) (y 1)))) #true)
  (test-equal (term (=`$\alpha$` (lambda (x) x) (lambda (y) z))) #false))

(define-metafunction SD
  =`$\alpha$` : e e -> boolean
  [(=`$\alpha$` e_1 e_2) ,(equal? (term (sd e_1)) (term (sd e_2)))])

(define (=`$\alpha$`/racket x y) (term (=`$\alpha$` ,x ,y)))

(module+ test
  (test-results))
\end{lstlisting}

\subsection{Substitution}

\textit{Substitution} is the syntactic equivalent of function application.

\begin{lstlisting}[frame=none,numbers=none]
; (subst ([e x] ...) e_*) substitutes e ... for x ... in e_* (hygienically)

(module+ test
  (test-equal (term (subst ([1 x][2 y]) x)) 1)
  (test-equal (term (subst ([1 x][2 y]) y)) 2)
  (test-equal (term (subst ([1 x][2 y]) z)) (term z))
  (test-equal (term (subst ([1 x][2 y]) (lambda (z w) (x y))))
              (term (lambda (z w) (1 2))))
  (test-equal (term (subst ([1 x][2 y]) (lambda (z w) (lambda (x) (x y)))))
              (term (lambda (z w) (lambda (x) (x 2))))
              #:equiv =`$\alpha$`/racket)
  (test-equal (term (subst ((2 x)) ((lambda (x) (1 x)) x)))
              (term ((lambda (x) (1 x)) 2))
              #:equiv =`$\alpha$`/racket))

(define-metafunction Lambda
  subst : ((any x) ...) any -> any
  [(subst [(any_1 x_1) ... (any_x x) (any_2 x_2) ...] x) any_x]
  [(subst [(any_1 x_1) ...]  x) x]
  [(subst [(any_1 x_1) ...]  (lambda (x ...) any_body))
   (lambda (x_new ...)
     (subst ((any_1 x_1) ...)
            (subst-raw ((x_new x) ...) any_body)))
   (where  (x_new ...)  ,(variables-not-in (term any_body) (term (x ...))))]
  [(subst [(any_1 x_1) ...]  (any ...)) ((subst [(any_1 x_1) ...]  any) ...)]
  [(subst [(any_1 x_1) ...]  any_*) any_*])

(define-metafunction Lambda
  subst-raw : ((x x) ...) any -> any
  [(subst-raw ((x_n1 x_o1) ... (x_new x) (x_n2 x_o2) ...) x) x_new]
  [(subst-raw ((x_n1 x_o1) ...)  x) x]
  [(subst-raw ((x_n1 x_o1) ...)  (lambda (x ...) any))
   (lambda (x ...) (subst-raw ((x_n1 x_o1) ...)  any))]
  [(subst-raw [(any_1 x_1) ...]  (any ...))
   ((subst-raw [(any_1 x_1) ...]  any) ...)]
  [(subst-raw [(any_1 x_1) ...]  any_*) any_*])

(module+ test
  (test-results))
\end{lstlisting}

\subsection{Reduction and semantics}

The logical way of generating an equivalence (or reduction) relation over terms
uses through inductive inference rules that make the relation compatible with
all syntactic constructions.

An alternative and equivalent method is to introduce the notion of a
\textit{context} and to use it to generate the reduction relation (or
equivalence) from the notion of reduction:

\begin{lstlisting}[frame=none,numbers=none]
(define-extended-language Lambda-calculus Lambda
  (e ::= .... n)
  (n ::= natural)
  (v ::= (lambda (x ...) e))

  ; a context is an expression with one hole in lieu of a sub-expression
  (C ::=
     hole
     (e ... C e ...)
     (lambda (x_!_ ...) C)))

(define Context? (redex-match? Lambda-calculus C))

(module+ test
  (define C1 (term ((lambda (x y) x) hole 1)))
  (define C2 (term ((lambda (x y) hole) 0 1)))
  (test-equal (Context? C1) #true)
  (test-equal (Context? C2) #true))

(module+ test
  (test-results))
\end{lstlisting}

\textit{Reduction relations}:

\begin{lstlisting}[frame=none,numbers=none]
; the `$\lambda \beta$` calculus, reductions only
(module+ test
  ; does the one-step reduction reduce both `$\beta$` redexes?
  (test--> -->`$\beta$`
           #:equiv =`$\alpha$`/racket
           (term ((lambda (x) ((lambda (y) y) x)) z))
           (term ((lambda (x) x) z))
           (term ((lambda (y) y) z)))

  ; does the full reduction relation reduce all redexes?
  (test-->> -->`$\beta$`
            (term ((lambda (x y) (x 1 y 2))
                   (lambda (a b c) a)
                   3))
            1))

(define -->`$\beta$`
  (reduction-relation
   Lambda-calculus
   (--> (in-hole C ((lambda (x_1 ..._n) e) e_1 ..._n))
   (in-hole C (subst ([e_1 x_1] ...) e)))))

(traces -->`$\beta$`
        (term ((lambda (x y)
                 ((lambda (f) (f (x 1 y 2)))
                  (lambda (w) 42)))
               ((lambda (x) x) (lambda (a b c) a))
               3)))
\end{lstlisting}

Defining the \textit{call-by-value} calculus requires just a small change to the
reduction rule:

\begin{lstlisting}[frame=none,numbers=none]
(define -->`$\beta$`v
  (reduction-relation
   Lambda-calculus
   (--> (in-hole C ((lambda (x_1 ..._n) e) v_1 ..._n))
        (in-hole C (subst ([v_1 x_1] ...) e)))))

(traces -->`$\beta$`v
        (term ((lambda (x y)
                 ((lambda (f) (f (x 1 y 2)))
                  (lambda (w) 42)))
               ((lambda (x) x) (lambda (a b c) a))
               3)))
\end{lstlisting}

\textit{Semantics}:

\begin{lstlisting}[frame=none,numbers=none]
(define-extended-language Standard Lambda-calculus
  (v ::= n (lambda (x ...) e))
  (E ::=
     hole
     (v ... E e ...)))

(module+ test
  (define t0
    (term
     ((lambda (x y) (x y))
      ((lambda (x) x) (lambda (x) x))
      ((lambda (x) x) 5))))
  (define t0-one-step
    (term
     ((lambda (x y) (x y))
      (lambda (x) x)
      ((lambda (x) x) 5))))

  ; yields only one term, leftmost-outermost
  (test--> s->βv t0 t0-one-step)
  ; but the transitive closure drives it to 5
  (test-->> s->βv t0 5))

(define s->βv
  (reduction-relation
   Standard
   (--> (in-hole E ((lambda (x_1 ..._n) e) v_1 ..._n))
        (in-hole E (subst ((v_1 x_1) ...) e)))))

(module+ test
  (test-results))
\end{lstlisting}

\begin{lstlisting}[frame=none,numbers=none]
(module+ test
  (test-equal (term (eval-value ,t0)) 5)
  (test-equal (term (eval-value ,t0-one-step)) 5)

  (define t1
    (term ((lambda (x) x) (lambda (x) x))))
  (test-equal (lambda? t1) #true)
  (test-equal (redex-match? Standard e t1) #true)
  (test-equal (term (eval-value ,t1)) 'closure))

(define-metafunction Standard
  eval-value : e -> v or closure
  [(eval-value e) any_1 (where any_1 (run-value e))])

(define-metafunction Standard
  run-value : e -> v or closure
  [(run-value n) n]
  [(run-value v) closure]
  [(run-value e)
   (run-value e_again)
   ; (v) means that we expect s->βv to be a function
   (where (e_again) ,(apply-reduction-relation s->βv (term e)))])

(module+ test
  (test-results))
\end{lstlisting}

\begin{thebibliography}{9}
\bibitem{redex-tutorial}
Robert Bruce Findler, Casey Klein, Burke Fetscher, and Matthias Felleisen.
(2015) \emph{Redex: Practical Semantics Engineering},
\url{https://docs.racket-lang.org/redex/index.html}.
\end{thebibliography}
