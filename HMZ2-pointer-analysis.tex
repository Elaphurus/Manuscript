\section{Pointer analysis}

When the lvalue of two objects coincides the objects are said to be aliased. In
languages with pointers and/or call-by-reference parameters, alias analysis is
the core part of most other data flow analyses. For example, live-variable
analysis of an experssion `*p = 13' must make worst-case assumptions without
alias information: `p' may reference all (visible) objects, which then
subsequently must be marked live. Clearly, this renders live-variable analysis
nearly useless. On the other hand, if it is known that only the aliases
$\{<*p, x>, <*p, y>\}$ are possible, only `x' and `y' need to be marked live.

A point-to map $[p \mapsto \{x, y\}]$ carries the same information as an alias
set, but it is a more compact representation.

\subsection{What makes C harder to analyze?}

\paragraph{Multi-level pointers and the address operator `\&'}
The address operator can be employed to compute the address of an object with an
lvalue. With only single-level pointers, the variable to be updated is
syntactically present in the expression. Call-by-reference and single-level
pointers can simulate multi-level pointers.

\paragraph{Structs and unions}
Pointers can point to members of structs.

\paragraph{Runtime memory allocations}
Pointers are allowed to point to both stack and heap allocated objects. Value
setting functoins may create a dynamic point-to relation, examples include
`p=\&x', `alloc()' and `strdup()'.

\paragraph{Type casts}
Type casts enable pointers to change type. A special characteristic of the C
languagee is that implementation-defined features are supported by the Standard.
A C program can comply to the Standard in two ways. A
\textit{strictly conforming} program shall not depend on implementation-defined
behavior but a \textit{conforming} program is allowed to do so. In ISO 1990,
cast of an integral value to a pointer or conversely is an
implementation-defined behavior. Cast of a pointer to a pointer with less
alignment requirement and back again, is strictly conforming.
Implementation-defined features cannot be described accurately by an
architecture-independent analysis.

\paragraph{Function pointers}
Pointers to functions can be defined.

\paragraph{Separate compilation}
A C program usually consists of a collection of translation units that are
compiled separately and linked to an executable. Suppose that a pointer analysis
is applied to a single module. This has two consequences. Potentially, global
variables may be modified by assignments in other modules. To be safe,
worst-case assumptions, i.e. Unknown, about global variables must be made.
Secondly, functions may be called from other modules with unknown parameters.
Thus, to be safe, all functions must be approximated by Unknown.

\subsection{Andersen's pointer analysis}

%% Andersen specifies the pointer analysis by the means of a non-standard type
%% inference system, which is related to the standard semantics. From the
%% specification, a constraint-based formulation is derived and an efficient
%% inference algorithm developed.

%% The set-based analysis consists of two parts: a specification and an inference
%% algorithm. The specification desribes the safety of a pointer approximation.
%% He presents a set of inference rules such that a pointer abstraction map
%% fulfills the rules only if the map is safe.

%% For every object of pointer type he determines a safe approximation to the set
%% of locations the pointer may contain during program execution, for all possible
%% input. A special case is function pointers. The result of the analysis is the
%% set of functions the pointer may invoke.

%% Constraint-based analysis resembles classical data-flow analysis, but has a
%% stronger semantical foundation.

A pointer abstraction is a map from abstract program objects (variables) to set
of abstract locations. An abstraction is safe if for every object of pointer
type, the set of concrete addresses it may contain at runtime is safely
described by the set of abstract locations.

\subsubsection{Abstract location}

A pointer is a variable containing the distinguished constant `NULL' or an
address. Due to casts, a pointer can (in principle) point to an arbitrary
address. An object is a set of logically related locations, e.g. four bytes
representing an integer value, or n bytes representing a struct value. Since
pointers may point to functions, we will also consider functions as objects.

An object can either be allocated on the program stack (local variables), at a
fixed location (strings and global variables), in the code space (functions), or
on the heap (runtime allocated objects). We shall only be concerned with the run
time allocated object brought into existence via `alloc()' calls. Assume that
all calls are labeled uniquely.

\begin{Definition}
  Let `p' be a pointer. If a pointer abstraction maps `p' to Unknown,
  $[p \mapsto Unknown]$, `p' may point to all accessible objects at runtime.
\end{Definition}

The unique abstract location Unknown denotes an arbitrary, unknown address,
which both be valid or illegal.

\begin{Definition}
  The set of abstract locations ALoc is defined inductively as follows:
  \begin{itemize}
  \item If v is the name of a global varible: $v \in ALoc$.
  \item If v is a parameter of some function with n variants:
    $v^i \in ALoc, i = 1, \dots, n$.
  \item If v is a local variable in some function with n variants:
    $v^i \in ALoc, i = 1, \dots, n$.
  \item If s is a string constant: $s \in ALoc$.
  \item If f is the name of a function with n variants:
    $f^i \in ALoc, i = 1, \dots, n$.
  \item If f is the name of a function with n variants:
    $f^i_0 \in ALoc, i = 1, \dots, n$.
  \item If l is the label of an alloc in a function with n variants:
    $l^i \in ALoc, i = 1, \dots, n$.
  \item If l is the label of an address operator in a function with n variants:
    $l^i \in ALoc, i = 1, \dots, n$.
  \item If $o \in ALoc$ denotes an object of type ``array'': $o[] \in ALoc$.
  \item If S is the type name of a struct or union type: $S \in ALoc$.
  \item If $S \in ALoc$ is of type ``struct'' or ``union'':
    $S.i \in ALoc$ for all fields i of S.
  \end{itemize}
  Names are assumed to be unique.
\end{Definition}

We assume that a program's static-call graph is available. The static-call graph
approximates the invocation of functoins, and assigns a variant number to
functions according to the call context. The location $f_0$ associated with
function $f$ denotes an abstract return location, i.e. a unique location where
$f$ delivers its return value. Arrays are treated as aggregates, that is, all
entries are merged. Fields of struct objects of the same name are merged, e.g.
given definition `struct S \{ int x; \} s, t', fields `s.x' and `t.x' are
collapsed.

% Point-to information for field members of a struct variable is associated with
% the definition of a struct; not the struct objects. For example, the point to
% information for member `s.x' is represented by `S1.x', where `S1' is the
% definition of `struct S1', which is a variant of the struct type.

\subsubsection{Pointer abstraction}

A pointer abstraction $\tilde{S} : ALoc \rightarrow \wp(ALoc)$ is a map from
abstract locations to sets of abstract locations.

\begin{Definition}
\label{def:pt-abs}
  A pointer abstraction $\tilde{S} : ALoc \rightarrow \wp(ALoc)$ is a map
  satisfying:
  \begin{enumerate}
  \item If $o \in ALoc$ is of base type: $\tilde{S}(o) = \{Unknown\}$.
  \item If $s \in ALoc$ is of struct/union type: $\tilde{S}(s) = \{\}$.
  \item If $f \in ALoc$ is a function designator: $\tilde{S}(f) = \{\}$.
  \item If $a \in ALoc$ is of type array: $\tilde{S}(a) = \{a[]\}$.
  \item $\tilde{S}(Unknown) = \{Unknown\}$.
  \end{enumerate}
\end{Definition}

The first condition requires that objects of base types are abstracted by
Unknown. The motivation is that the value may be cast into a pointer, and is
hence Unknown (in general). The fourth condition requires that an array
variable points to the content. In reality, `a' in `a[10]' is not a lvalue. It
is, however, convenient to consider `a' to be a pointer to the content.

\subsubsection{Safe pointer abstraction}

A pointer abstraction for a program is safe if for all input, every object a
pointer may point to at runtime is captured by the abstraction.

Let the abstraction functoin $\alpha : Loc \rightarrow ALoc$ be defined the
obvious way. An execution path from the initial program point $p_0$ and an
inital program store $S_0$ is denoted by
$$<p_0,S_0> \rightarrow \dots \rightarrow <p_n,S_n>$$
where $S_n$ is the store at program point $p_0$.

Let $p$ be a program and $S_0$ an initial store (mapping the program input to
the parameters of the goal function). Let $p_n$ be a program point, and
$\mathcal{L}_n$ the locations of all visible variables. A pointer abstraction
$\tilde{S}$ is safe with respect to $p$ if
$$l \in \mathcal{L}_n : \alpha(S_n(l)) \subseteq \tilde{S}(\alpha(l))$$
whenever $<p_0,S_0> \rightarrow \dots \rightarrow <p_n,S_n>$.

The definition of safety above considers only monolithic programs where no
external functions nor variables exist. We are, however, interested in analysis
of translation units where parts of the program may be undefined.

\begin{Definition}
  Let $p \equiv m_1, \dots, m_m$ be a program consisting of the modules $m_i$.
  A pointer abstraction $\tilde{S}$ is safe for $m_{i0}$ if for all program
  points $p_n$ and initial stores $S_0$ where
  $<p_0,S_0> \rightarrow \dots \rightarrow <p_n,S_n>$, then:
  \begin{itemize}
  \item for
    $l \in \mathcal{L}_n : \alpha(S_n(l)) \subseteq \tilde{S}(\alpha(l))$ if $l$
    is defined in $m_{i0}$,
  \item for $l \in \mathcal{L}_n : \tilde{S}(l) = \{Unknown\}$ if $l$ is defined
    in $m_i \neq m_{i0}$
  \end{itemize}
  where $\mathcal{L}_n$ is the set of visible variables at program point $n$.
\end{Definition}

\subsubsection{Pointer analysis specification}

We specify a flow-insensitve (summary), intra-procedural pointer analysis.

The specification can be employed to check that a given pointer abstraction
$\tilde{S}$ is safe for a program. The specification is in the form of inference
rules $\tilde{S} \vdash p : \bullet$.

We argue that if the program fulfills the rules in the context of a pointer
abstraction $\tilde{S}$, then $\tilde{S}$ is a safe pointer abstraction.
Actually, the rules will also fail if $\tilde{S}$ is not a pointer abstraction,
i.e. does not satisfy \autoref{def:pt-abs}. Let $\tilde{S}$ be given.

Suppose that $d \equiv x : T$ is a definition (i.e., not an `extern'
declaration). The safety of $\mathcal{S}$ with respect to $d$ depends on the
type $T$.

\begin{Lemma}
  Let $d \in Decl$ be a definition. Then
  $\tilde{S} : ALoc \rightarrow \wp(ALoc)$ is a pointer abstraction with respect
  to $d$ if
  $$\tilde{S} \vdash^{pdecl} d : \bullet$$
  where $\vdash^{pdecl}$ is defined in , and $\tilde{S}(Unknown) = \{Unknown\}$.
\end{Lemma}
